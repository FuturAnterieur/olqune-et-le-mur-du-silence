\documentclass[a4paper,11pt, openany]{book}
\usepackage[french]{babel}
\usepackage[T1]{fontenc}
\usepackage[utf8]{inputenc}
\usepackage{lmodern}
\usepackage{microtype}
\usepackage{hyperref}

\title{Olqune et le mur du silence}
\author{Futur Antérieur}
\date{}

\begin{document}

{\Large Olqune et le mur du silence}

\chapter{L'attente}
Olqune, preux sorcier de l'hyper-espace, scrutait attentivement le cosmos par la baie vitrée du poste de commandement de son vaisseau.
Il était quelque peu fébrile, ou en tout cas plus qu'à son habitude, car il ne voulait pas manquer le passage d'une comète bleue encore inconnue, tel que l'astrologue l'avait prédit à la radio.
Il concentrait ainsi toute son attention sur le vide spatial. Au bout d'environ 15 minutes, il se lassa de cet exercice, se leva 
et alla se préparer une tisane au comptoir à thé, juste à côté de la chute à glace du vaisseau (son vaisseau fonctionnait en effet à la glace).

Sirotant sa tisane, il eut un instant de réflexion. Il réalisa qu'en vérité, la comète lui importait peu.
En vérité, il s'inquiétait du silence de son grand amour, Iled le furet, qui avait laissé son dernier message sans réponse, et il cherchait des manières plus simples d'être anxieux.

Maintenant, il se sentait moribond. Mais il n'avait pas envie de retourner à sa baie vitrée. Il n'avait envie que d'attendre et de penser au passé.

\chapter{Le passé}
La forêt regorgeait de végétation et de mystères. Une rivière de printemps coulait dans la lumière du matin, chevauchée par un pont scintillant de pierre vigoureuse.
Le chevalier et la magicienne le traversaient sans se hâter, savourant pleinement leur solitude abondante. 
De l'autre côté, les sentiers bifurquaient sans cesse à l'ombre des arbres, mais ils s'y engagèrent sans hésitation, suivant un chemin qu'eux seuls pouvaient voir.
Alors que la pénombre s'immisçait à travers les frondaisons, ils parvinrent à une clairière où se dressait un sanctuaire antique surplombé d'un dais aux colonnes de marbre. 

La magicienne fit un signe de la main au chevalier, lui indiquant qu'ils avaient atteint leur but. 
Ils gravirent ensemble les marches du sanctuaire, y trouvant, sur un présentoir de bois fatigué, un grimoire refermé, à la couverture muette.
C'était le Mystère des Sept Consulats, la synthèse des sagesses des âges du passé, la piste du savoir de l'au-delà.

Le chevalier prit le grimoire et le passa à la magicienne, qui se mit à parcourir ses feuillets usés par les années. Après un moment, ses yeux s'assombrirent. Elle dit :
"Le signe des temps nous a menti. Ce n'est pas le livre que nous cherchions."

La magicienne cacha le livre sous sa cape de toile, puis son image s'embruma et s'évanouit dans la nuit qui enveloppait la forêt.
Le chevalier fut pris d'une vive inquiétude, réalisant qu'il avait été la proie d'une illusion. 
Sans la présence de la magicienne, il aurait du mal à retracer le chemin l'ayant mené jusqu'au sanctuaire. 
Comme en réponse à la disparition du livre, il pouvait maintenant sentir les esprits de la nuit se tordre et se déformer, 
parcourus des douloureux spasmes de la faim d'âmes humaines. 

Sentant une peur monter en lui, et craignant qu'elle n'en vienne qu'à s'amplifier, 
il sortit du sanctuaire et entreprit de quitter la forêt.
Il pensait que s'il suivait la ligne la plus droite possible à travers les sentiers, il parviendrait bien par en trouver la sortie.
Mais au détour d'un énième chemin plongé dans l'ombre, il s'enfargea dans une racine et tomba à la renverse.
Il sentait un esprit de la nuit derrière lui. En se retournant, il le vit : c'était une chose indistincte, fondue dans l'obscurité, 
qui rampait erratiquement mais qui, pourtant, se rapprochait inéluctablement. 

Il tenta de se relever pour reprendre sa course, mais son pied s'était coincé dans la racine, et un instant de trop s'écoula : l'esprit fut sur lui.
Un froid perçant se glissa sous sa peau, contre lequel il se crispa. Mais il avait beau se cabrer ainsi contre la présence ennemie et tenter de la rejeter de son corps, rien n'y faisait.
L'esprit s'immisçait toujours plus profondément en lui, mû par une force dépourvue de volonté, aussi naturelle que les flots d'une rivière.
Avec crainte, il entrevit son âme dévorée par la nuit, la fin de sa conscience. Il deviendrait un être habité d'une vie qu'il ne pourrait observer.
Il songea qu'il n'avait jamais vu la mort approcher auparavant.

Un temps indistinct s'écoula. Le chevalier avait l'impression d'être suspendu dans un vide d'un noir total. Mais il pouvait percevoir comme une voix muette, une énonciation de mots en lui.
Ces mots disaient : "Tu as la clé du souffle, j'ai la clé de l'au-delà. Nous ne sommes plus qu'un, affranchis du passage du temps, immortels."
Alors que le vide s'animait d'une vague de chaleur enveloppante, le chevalier réalisa qu'il n'était pas inconscient. 
Il pensa alors à la magicienne, et la voix lui dit : "Pour la rejoindre, nous franchirons ensemble le mur du silence."
Ces affirmations éveillaient en lui un fouillis de questions, mais il n'arrivait pas à les démêler, et il n'aurait pas non plus osé répondre directement à la voix.
Il se rappela ainsi son vif désir d'échapper à la forêt et ses dangers, et la voix le rassura : "Je suis la voix de l'histoire, et nous avons banni la nuit. Tu n'auras qu'à ouvrir les yeux pour le voir, quand le temps sera venu."

Le chevalier sut alors qu'il dormait. Il voguait sur un grand bateau, avec une myriade d'autres passagers. 
Un tyran ayant perdu son coeur avait tenté de détourner le bateau, mais une intervention divine l'en avait empêché.
Le chevalier parla avec le tyran, qui lui avoua qu'il ne savait plus dans quelle direction allait sa vie.   

Quand le chevalier se réveilla, il était toujours étendu sur le sentier où il était tombé, et les rayons gris de l'aurore
baignaient son lit de terre et de feuilles. Il se désemcombra de la racine ayant causé sa chute, et il se mit en route vers le pont à l'orée de la forêt.
Même si la magicienne n'était plus avec lui, il chemina avec assurance dans les sentiers, retraçant le parcours de la veille.
À la lisière des arbres, la rivière s'avançait dans la plaine et menait jusqu'à la ville de Cumes, avec sa muraille, son clocher et ses petits toits pentus ponctuant l'horizon. 
Le chevalier avait faim, d'une faim dont il attendait le soulagement avec une douce espérance.

\chapter{La conspiration}

Olqune était curieux de savoir ce qu'il était advenu de la comète dont il avait manqué le passage, et il syntonisa donc son sa radio pour tomber sur l'émission d'Orion Poisson, son astrologue préféré.
Il disait : 

\begin{quote}
...en effet, comme certains d'entre vous ont pu le voir, la comète que j'avais annoncée est bel et bien passée. Mais il y a plus important. Comme j'ai pu le prédire par l'observation de la Lune en Scorpion,
prédiction accessoirement confirmée par le musée du livre de la ville de Cumes, le fameux Mystère des Sept Consulats, véritable grimoire de toutes les sagesses anciennes, sera exposé dans ce musée jusqu'à la semaine prochaine.
Je recommande fortement à tous mes auditeurs et auditrices d'aller y jeter un coup d'oeil. Ils y trouveront peut-être des vérités insoupçonnées, quoique pas autant qu'en écoutant mon émission. J'ai même une amie qui travaille à Cumes, d'ailleurs.
Mais bon, trêve de bavardages, passons au sujet croustillant de l'horoscope des résultats sportifs de la semaine à venir...
\end{quote}

Olqune éteignit la radio à ce moment. L'horoscope sportif l'intéressait peu. 
"Comme je suis pitoyable", se disait-il. "Je ne cherche que des baumes éphémères à la plaie béante de mon ennui éternel. Je ne serai jamais pleinement heureux."
Il se rendit compte qu'il était maintenant d'humeur méditative, alors il pencha son siège de pilotage et ferma les yeux.
Il eut alors une vision. 

C'était Ra, l'un des dieux solaires égyptiens, avec sa tête de faucon et un soleil sur la tête. Ra lui parla des sept niveaux de conscience, qu'on pouvait aussi bien traverser dans une vie que dans toute une journée.
Il lui montra ensuite un livre très étrange, dont les pages avaient la couleur d'un arc-en-ciel dans le désordre, suivant le chapitre auquel elles appartenaient. Olqune ne parvint pas à lire le titre du livre, ni à en garder en mémoire ne serait-ce qu'un passage,
mais il réalisa que d'un chapitre à l'autre, une thèse était creusée de plus en plus profondément, cette thèse exposant en détail en quoi le monde moderne était inexorablement voué à la misère. 
Sentant que la vision tirait à sa fin, il remercia Ra de ce cadeau.

Olqune ouvrit les yeux en se disant qu'une visite au musée pourrait certainement lui changer les idées. Il mit le cap sur la planète Terre, où se trouvait Cumes. La Terre était à 15 années-lumière de son vaisseau, mais comme 
ce dernier voyageait à la vitesse de la pensée, il pourrait s'y rendre en quelques secondes sans problème. 

\chapter{L'oracle}

C'étaient les petites heures du dimanche matin dans les rues de Cumes. Baignées d'une brume légère, les maisons endormies se blotissaient les unes contre les autres sur leur lit de pavé gris.
Parmi elles, des gardes en armure légère terminaient leur ronde de nuit, saluant avec soulagement la venue de l'aurore à travers les nuages violets comme esquissés par un peintre dans le ciel.   

L'un de leurs camarades, assigné pour sa part à la tour de guet des portes de la ville, vit apparaître à l'horizon la silhouette d'un homme s'avançant sur la plaine.
Le garde maintint son regard sur l'inconnu, curieux de savoir où pouvait bien aller ce voyageur solitaire; il semblait bien qu'il se dirigeasse vers Cumes. 
En effet, l'inconnu se planta devant la porte et s'adressa à la personne qui en était manifestement responsable, c'est-à-dire le sus-nommé garde de la tour de guet.


"Holà, puis-je entrer" héla le voyageur.

"Probablement. Quelle est la raison de votre visite?" Le garde entamait son protocole habituel.

- Je suis un chevalier errant, en quête d'un gîte de passage.

Le chevalier en question ne portait qu'un simple vêtement de paysan, une tunique et des haut-de-chausses d'un beige indistinct. Il n'avait même pas d'épée.

- Ah bon, un chevalier? Quel serait donc votre domaine?

- Cela n'est pas important. On pourrait dire Lyonesse, par exemple. Mais mon réel domaine n'est connu que de moi-même.

- Euhm, je vois. Écoutez, les portes ouvrent de toute façon lorsque le clocher sonne prime, alors vous pourrez entrer à ce moment.

- J'imagine que je n'ai pas le choix. Très bien alors!

Au cours de l'heure qui suivit, le garde maintint un oeil protocolaire quoique récalcitrant sur le chevalier, ce dernier partageant son temps entre de furtifs parcours panoramiques le long du ruisseau voisin et du pur et simple poireautage en face des lourdes portes de chêne.
Lorsqu'enfin ces dernières s'ouvrirent, il fit un signe de la main enjoué au garde, qui l'ignora d'ailleurs sans trop de surprise, et il pénétra dans les rues de Cumes, qui commençaient déjà à s'animer.
Plusieurs citadins remarquèrent le passage de cet étranger, surtout à cause de la fixité singulière de son regard et de son obstination visible à ne pas demander son chemin malgré son besoin évident d'orientation.
Errant d'un tronçon de rue à l'autre toute la matinée, et comme guidé par une intuition à la source houleuse, il atteignit éventuellement la taverne locale, où il se reposa pour quelques jours.

Au cours de ce temps de repos, le chevalier prit l'habitude de prendre des marches quotidiennes dans Cumes. Au hasard de ces promenades, il put graduellement établir une carte mentale rudimentaire des lieux qui lui semblaient significatifs. 
Il avait ainsi remarqué une curieuse petite maison appuyée sur le flanc gauche de l'austère église de la ville. La maison ne possédait pas de fenêtres et n'avait qu'un grande portique double en bois, ceint d'une arche de pierre. Elle semblait en fait dater de l'Antiquité,
et le chevalier se plaisait à s'imaginer qu'il s'agissait réellement du plus vieil édifice de Cumes. Un mardi matin, alors qu'il repassait devant la maison, il vit ses portes entrouvertes et une vieille dame bossue en train de balayer sa devanture.
La dame avait de longs cheveux gris filandreux et des rides à travers lesquels s'apercevaient les vestiges de traits harmonieux. Elle ne portait qu'une simple robe dont il était difficile de distinguer la couleur à travers les couches de poussière qui la recouvraient. 
Son balai aux poils hérissés caressait maladroitement le sol par mouvements saccadés.

Le chevalier avait bien l'intention de passer son chemin, mais lorsque la vieille femme l'aperçut, un éclair de compréhension passa dans ses grands yeux noirs.

"Auriez-vous perdu votre chemin, cher voyageur? Je crois que je peux vous aider." lui dit-elle d'une voix avenante.

Le chevalier fut d'abord pris au dépourvu. Mais, dans sa tête, la voix de l'histoire guida ses paroles : 

- Ce serait bien difficile, car pour l'instant, je n'ai pas de destination - je ne fais qu'une promenade. À qui ai-je l'honneur, d'ailleurs?

- Je suis la Sibylle du village, cher monsieur. Les gens me consultent pour leurs problèmes d'avenir et d'esprit. 
Et, si je peux me permettre, nous marchons toujours vers une destination. Par où êtes-vous passé pour arriver ici? Si je devais deviner, je dirais...

Elle marqua une pause.

- Je dirais la forêt voisine. Est-ce que je me trompe?

Le chevalier fut franc.

- Non, en fait, c'était bel et bien une étape de mon parcours.

- Alors vous devez être fatigué. Allez, je vous invite à prendre le thé chez moi. Vous aurez tout le loisir de terminer votre errance après, n'est-ce-pas?

- Euhm, oui, je suppose...

Et le chevalier marcha à la suite de la Sibylle dans sa demeure, alors que la voix de l'histoire avait décidé de se taire pour cet instant. 
Passant le grand portique de bois, ils se trouvèrent dans un salon où, seuls entre les murs gris, 
deux fauteuils étaient placés de biais de chaque côté d'une tablette de bois où trônait une théière entourée de ses soucoupes. L'austérité du décor fit naître une impression de sérénité dans le coeur du chevalier.
La Sibylle prit place dans le fauteuil le plus éloigné.
"Vous pouvez vous joindre à moi. C'est ici que je tiens mes consultations, d'habitude."

Le chevalier s'assit à son tour, plaçant les mains sur ses genoux et promenant son regard dans la pièce. La Sibylle continua la conversation.

- Comme j'y pense, vous ne vous êtes pas encore présenté. Quel est votre nom?

Le chevalier marqua une pause. La voix de l'histoire persistait dans son silence.

- Je me nomme Tristan. Sire Tristan, pour vous servir.




\end{document}