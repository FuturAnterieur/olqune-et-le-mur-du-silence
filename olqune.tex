\documentclass[a4paper,11pt, openany]{book}
\usepackage[french]{babel}
\usepackage[T1]{fontenc}
\usepackage[utf8]{inputenc}
\usepackage{lmodern}
\usepackage{microtype}
\usepackage{hyperref}

\title{Olqune et le mur du silence}
\author{Futur Antérieur}

\begin{document}

\maketitle

\chapter{L'attente}
Olqune, preux sorcier de l'hyper-espace, scrutait attentivement le cosmos par la baie vitrée du poste de commandement de son vaisseau.
Il était quelque peu fébrile, ou en tout cas plus qu'à son habitude, car il ne voulait pas manquer le passage d'une comète bleue encore inconnue, tel que son astrologue l'avait prédit.
Il concentrait ainsi toute son attention sur le vide spatial. Au bout d'environ 15 minutes, il se lassa de cet exercice, se leva 
et alla se préparer une tisane au comptoir à thé, juste à côté de la chute à glace du vaisseau (son vaisseau fonctionnait en effet à la glace).

Sirotant sa tisane, il eut un instant de réflexion. Il réalisa qu'en vérité, la comète lui importait peu.
En vérité, il s'inquiétait du silence de son amie, Iled le furet, qui avait laissé son dernier message sans réponse, et il cherchait des manières plus simples d'être anxieux.

Maintenant, il se sentait moribond. Mais il n'avait pas envie de retourner à sa baie vitrée. Il n'avait envie que d'attendre et de penser au passé.

\chapter{Le passé}
La forêt regorgeait de végétation et de mystères. Une rivière de printemps coulait dans la lumière du matin, chevauchée par un pont scintillant de pierre vigoureuse.
Le chevalier et la magicienne le traversaient sans se hâter, savourant pleinement leur solitude abondante. 
De l'autre côté, les sentiers bifurquaient sans cesse à l'ombre des arbres, mais ils s'y engagèrent sans hésitation, suivant un chemin qu'eux seuls pouvaient voir.
Alors que la pénombre s'immiscait à travers les frondaisons, ils parvinrent à une clairière où se dressait un sanctuaire antique surplombé d'un dais aux colonnes de marbre. 

La magicienne fit un signe de la main au chevalier, lui indiquant qu'ils avaient atteint leur but. 
Ils gravirent ensemble les marches du sanctuaire, y trouvant, sur un présentoir de bois fatigué, un grimoire refermé, à la couverture muette.
C'était le Mystère des Sept Consulats, la synthèse des sagesses des âges du passé, la piste du savoir de l'au-delà.

Le chevalier prit le grimoire et le passa à la magicienne, qui se mit à parcourir ses feuillets usés par les années. Après un moment, ses yeux s'assombrirent. Elle dit :
"Le signe des temps nous a menti. Ce n'est pas le livre que nous cherchions."

La magicienne cacha le livre sous sa cape de toile, puis son image s'embruma et s'évanouit dans la nuit qui enveloppait la forêt.
Le chevalier fut pris d'une vive inquiétude, réalisant qu'il avait été la proie d'une illusion. 
Il s'aperçut qu'il avait oublié le chemin l'ayant mené à travers les sentiers jusqu'au sanctuaire. 
Dans l'obscurité environnante, il pouvait sentir les esprits de la nuit se tordre et se déchirer, 
parcourus des douloureux spasmes de la faim d'âmes humaines.

Sentant une peur monter en lui, et craignant qu'elle n'en vienne qu'à s'amplifier, 
il sortit du sanctuaire et entreprit de revenir sur ses pas dans le sentier qui l'avait mené à la clairière.
Il pensait que s'il courait sans s'arrêter en ligne droite, il parviendrait bien par trouver une sortie à la forêt.
Mais au détour d'un autre chemin plongé dans l'ombre, il s'enfargea dans une racine cachée par un buisson et tomba à la renverse.
Il sentait un esprit de la nuit derrière lui. En se retournant, il le vit : c'était une chose indistincte, vivante, qui rampait erratiquement mais qui se rapprochait inéluctablement d'un instant à l'autre.









\end{document}