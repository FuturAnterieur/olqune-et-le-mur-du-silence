\documentclass[a4paper,11pt, openany]{book}
\usepackage[french]{babel}
\usepackage[T1]{fontenc}
\usepackage[utf8]{inputenc}
\usepackage{lmodern}
\usepackage{microtype}
\usepackage{hyperref}

\title{Olqune et le mur du silence}
\author{Futur Antérieur}

\begin{document}

\maketitle

\chapter{L'attente}
Olqune, preux sorcier de l'hyper-espace, scrutait attentivement le cosmos par la baie vitrée du poste de commandement de son vaisseau.
Il était quelque peu fébrile, ou en tout cas plus qu'à son habitude, car il ne voulait pas manquer le passage d'une comète bleue encore inconnue, tel que l'astrologue l'avait prédit à la radio.
Il concentrait ainsi toute son attention sur le vide spatial. Au bout d'environ 15 minutes, il se lassa de cet exercice, se leva 
et alla se préparer une tisane au comptoir à thé, juste à côté de la chute à glace du vaisseau (son vaisseau fonctionnait en effet à la glace).

Sirotant sa tisane, il eut un instant de réflexion. Il réalisa qu'en vérité, la comète lui importait peu.
En vérité, il s'inquiétait du silence de son amie, Iled le furet, qui avait laissé son dernier message sans réponse, et il cherchait des manières plus simples d'être anxieux.

Maintenant, il se sentait moribond. Mais il n'avait pas envie de retourner à sa baie vitrée. Il n'avait envie que d'attendre et de penser au passé.

\chapter{Le passé}
La forêt regorgeait de végétation et de mystères. Une rivière de printemps coulait dans la lumière du matin, chevauchée par un pont scintillant de pierre vigoureuse.
Le chevalier et la magicienne le traversaient sans se hâter, savourant pleinement leur solitude abondante. 
De l'autre côté, les sentiers bifurquaient sans cesse à l'ombre des arbres, mais ils s'y engagèrent sans hésitation, suivant un chemin qu'eux seuls pouvaient voir.
Alors que la pénombre s'immisçait à travers les frondaisons, ils parvinrent à une clairière où se dressait un sanctuaire antique surplombé d'un dais aux colonnes de marbre. 

La magicienne fit un signe de la main au chevalier, lui indiquant qu'ils avaient atteint leur but. 
Ils gravirent ensemble les marches du sanctuaire, y trouvant, sur un présentoir de bois fatigué, un grimoire refermé, à la couverture muette.
C'était le Mystère des Sept Consulats, la synthèse des sagesses des âges du passé, la piste du savoir de l'au-delà.

Le chevalier prit le grimoire et le passa à la magicienne, qui se mit à parcourir ses feuillets usés par les années. Après un moment, ses yeux s'assombrirent. Elle dit :
"Le signe des temps nous a menti. Ce n'est pas le livre que nous cherchions."

La magicienne cacha le livre sous sa cape de toile, puis son image s'embruma et s'évanouit dans la nuit qui enveloppait la forêt.
Le chevalier fut pris d'une vive inquiétude, réalisant qu'il avait été la proie d'une illusion. 
Sans la présence de la magicienne, il aurait du mal à retracer le chemin l'ayant mené jusqu'au sanctuaire. 
Comme en réponse à la disparition du livre, il pouvait maintenant sentir les esprits de la nuit se tordre et se déformer, 
parcourus des douloureux spasmes de la faim d'âmes humaines. 

Sentant une peur monter en lui, et craignant qu'elle n'en vienne qu'à s'amplifier, 
il sortit du sanctuaire et entreprit de quitter la forêt.
Il pensait que s'il suivait la ligne la plus droite possible à travers les sentiers, il parviendrait bien par en trouver la sortie.
Mais au détour d'un énième chemin plongé dans l'ombre, il s'enfargea dans une racine et tomba à la renverse.
Il sentait un esprit de la nuit derrière lui. En se retournant, il le vit : c'était une chose indistincte, fondue dans l'obscurité, 
qui rampait erratiquement mais qui, pourtant, se rapprochait inéluctablement. 

Il tenta de se relever pour reprendre sa course, mais son pied s'était coincé dans la racine, et un instant de trop s'écoula : l'esprit fut sur lui.
Un froid perçant se glissa sous sa peau, contre lequel il se crispa. Mais il avait beau se cabrer ainsi contre la présence ennemie et tenter de la rejeter de son corps, rien n'y faisait.
L'esprit s'immisçait toujours plus profondément en lui, mû par une force dépourvue de volonté, aussi naturelle que les flots d'une rivière.
Avec crainte, il entrevit son âme dévorée par la nuit, la fin de sa conscience. Il deviendrait un être habité d'une vie qu'il ne pourrait observer.
Il songea qu'il n'avait jamais aperçu la mort approcher auparavant.

Un temps indistinct s'écoula. Le chevalier avait l'impression d'être suspendu dans un vide d'un noir total. Mais il pouvait percevoir comme une voix muette, une énonciation de mots en lui.
Ces mots disaient : "Tu as la clé du souffle, j'ai la clé de l'au-delà. Nous ne sommes plus qu'un, affranchis du passage du temps, immortels."
Alors que le vide s'animait d'une vague de chaleur enveloppante, le chevalier réalisa qu'il n'était pas inconscient. 
Il pensa alors à la magicienne, et la voix lui dit : "Elle nous attend de l'autre côté du mur du silence. Ensemble, nous le franchirons."
Ces affirmations éveillaient en lui un fouillis de questions, mais il n'arrivait pas à les démêler, et il n'aurait pas non plus osé répondre directement à la voix.
Il se rappela ainsi son vif désir d'échapper à la forêt et ses dangers, et la voix le rassura : "Je suis ta conscience, et nous avons banni la nuit. Tu n'auras qu'à ouvrir les yeux pour le voir, quand le temps sera venu."

Le chevalier sut alors qu'il dormait. Il voguait sur un grand bateau, avec une myriade d'autres passagers. 
Un tyran ayant perdu son coeur avait tenté de détourner le bateau, mais une intervention divine l'en avait empêché.
Le chevalier parla avec le tyran, qui lui avoua qu'il ne savait plus dans quelle direction allait sa vie.   

Quand le chevalier se réveilla, il était toujours étendu sur le sentier où il était tombé, et les rayons du soleil
baignaient son lit de terre et de feuilles. Il se désemcombra de la racine ayant causé sa chute, et il se mit en route vers le pont à l'orée de la forêt.
Même si la magicienne n'était plus avec lui, il chemina avec assurance dans les sentiers, retraçant le parcours de la veille.
À la lisière des arbres, la rivière s'avançait dans la plaine et menait jusqu'à la ville de Cumes, avec sa muraille, son clocher et ses petits toits pentus ponctuant l'horizon. 
Le chevalier avait faim, d'une faim dont il attendait le soulagement avec une douce espérance.

\chapter{La conspiration}

Olqune était curieux de savoir ce qu'il était advenu de la comète dont il avait manqué le passage, et il syntonisa donc son sa radio pour tomber sur l'émission d'Orion Poisson, son astrologue préféré.
Il disait : 

\begin{quote}
...en effet, comme certains d'entre vous ont pu le voir, la comète que j'avais annoncée est bel et bien passée. Mais il y a plus important. Comme j'ai pu le prédire par l'observation de la Lune en Scorpion,
prédiction accessoirement confirmée par le musée du livre de la ville de Cumes, le fameux Mystère des Sept Consulats, véritable grimoire de toutes les sagesses anciennes, sera exposé dans ce musée jusqu'à la semaine prochaine.
Je recommande fortement à tous mes auditeurs et auditrices d'aller y jeter un coup d'oeil. Ils y trouveront peut-être des vérités insoupçonnées, quoique pas autant qu'en écoutant mon émission. J'ai même une amie qui travaille à Cumes, d'ailleurs.
Mais bon, trêve de bavardages, passons au sujet pressant de l'horoscope des résultats sportifs de la semaine à venir...
\end{quote}

Olqune éteignit la radio à ce moment. L'horoscope sportif l'intéressait peu. 
"Comme je suis pitoyable", se disait-il. "Je ne cherche que des baumes éphémères à la plaie béante de mon ennui éternel. Je ne serai jamais pleinement heureux."
Il se rendit compte qu'il était maintenant d'humeur méditative, alors il pencha son siège de pilotage et ferma les yeux.
Il eut alors une vision. 

C'était Ra, l'un des dieux solaires égyptiens, avec sa tête de faucon et un soleil sur la tête. Ra lui parla des sept niveaux de conscience, qu'on pouvait aussi bien traverser dans une vie que dans toute une journée.
Il lui montra ensuite un livre très étrange, dont les pages avaient la couleur d'un arc-en-ciel dans le désordre, suivant le chapitre auquel elles appartenaient. Olqune ne parvint pas à lire le titre du livre, ni à en garder en mémoire ne serait-ce qu'un passage,
mais il réalisa que d'un chapitre à l'autre, une thèse était creusée de plus en plus profondément, cette thèse exposant en détail en quoi le monde moderne était inexorablement voué à la misère. 
Sentant que la vision tirait à sa fin, il remercia néanmoins Ra de ce cadeau.

Olqune ouvrit les yeux en se disant qu'une visite au musée pourrait certainement lui changer les idées. Il mit le cap sur la planète Terre, où se trouvait Cumes. La Terre était à 15 années-lumière de son vaisseau, mais comme 
ce dernier voyageait à la vitesse de la pensée, il pourrait s'y rendre en quelques secondes sans problème. 

\end{document}